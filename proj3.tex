\documentclass[11pt, a4paper]{article}

\usepackage[utf8]{inputenc}
\usepackage[top=3cm, left=2cm, text={17cm, 24cm}]{geometry}
\usepackage[czech]{babel} 
\usepackage{graphics}
\usepackage[ruled, czech, linesnumbered, noline, longend]{algorithm2e}
\usepackage{caption}
\usepackage{stackengine}
\usepackage{times}
\usepackage{multirow}
\usepackage{pdflscape}
\usepackage{graphicx}
\usepackage{subcaption}
\usepackage{hyperref}


%================================================== UVOD ==========================================================%

\begin{document}

    \catcode`\-=12

    \begin{titlepage}
        \begin{center}
            {\Huge \textsc{Vysoké učení technické v~Brně}} \\
            \vspace{\stretch{0.015}}
            {\huge \textsc
                {Fakulta informačních technologií}}\\
            
            \vspace{\stretch{0.4}}
            \LARGE
            {Typografie a~publikování\,--\,3.~projekt} \\
            \Huge{
            Tabulky a~obrázky}\\
            \vspace{\stretch{0.6}}
    \end{center}

    {\Large {\today \hfill Maroš Geffert (xgeffe00)}}
\end{titlepage}

%================================================ Tabulky ====================================================%

\section{Úvodní strana}
Název práce umístěte do zlatého řezu a~nezapomentě uvést dnešní dátum a~vaše a~příjmení.


\section{Tabulky}
Pro sázení tabulek můžeme použít buď prostředí \texttt{tabbing} nebo prostředí \texttt{tabular}.

\subsection{Prostředí \texttt{tabbing}}
Pri použití \texttt{tabbing} vzpadá tabulka následovně:

%---------------------------------------------------------------------------------
	\begin{tabbing}
		Vodní melouny   \quad\= \textbf{Cena} \quad\= \textbf{Množství}	    \kill
		\textbf{Ovoce}		\> \textbf{Cena}		\> \textbf{Množství}	\\
		Jablka				\> 25,90				\> 3 kg					\\
		Hrušky				\> 27,40				\> 2,5 kg				\\
		Vodní melouny		\> 35,--				\> 1 kus				\\
	\end{tabbing}
%---------------------------------------------------------------------------------
Toto prostředí se dá také použíť pro sázení algoritmu, ovšem vhodnejší je použít prostředí \texttt{algorithm} nebo \texttt{algorithm2e} (viz sekce~\ref{section:algoritmy}).

\subsection{Prostředí \texttt{tabular}}
Další možností, jak vytvořit tabulku, je použít prostředí \texttt{tabular}. Tabulky pak budou vypadat takto\footnote{Kdyby byl problem s \texttt{cline}, zkuste se podivat treba sem: \href{http://www.abclinuxu.cz/tex/poradna/show/325037}{http://www.abclinuxu.cz/tex/poradna/show/325037}.}:
\bigskip

\begin{table}[h]
    \centering
    \begin{tabular}{| l | l | l |}
        \hline
%--------------------------------------------------------------------
    & \multicolumn{2}{c |}{\textbf{Cena}} \\ \cline{2-3}
    \textbf{Měna}	& \textbf{nákup}	& \textbf{prodej}  \\ \hline
            EUR				& 25,615				& 27,20	\\
			GBP				& 29,899				& 31,80	\\
			USD				& 22,571 				& 25,51	\\ \hline
%--------------------------------------------------------------------

\end{tabular}
\caption{Tabulka kurzů k~dnešnímu dni}
\label{table:mena}

\end{table}
\bigskip

\begin{table}[h]
\centering
\begin{tabular}[p]{| l | l |}
    \hline
    $A$       & ${\neg}A$ \\ 
    \hline
    \textbf{P}  & N \\ \hline
    \textbf{O}  & O \\ \hline
    \textbf{X}  & X \\ \hline
    \textbf{N}  & P \\ \hline
\end{tabular} 
\begin{tabular}[p]{| l | l | l | l | l | l |}
\hline
\multicolumn{2}{| l |}{\multirow{2}{*}{$ A \wedge B $}} & \multicolumn{4}{ c |}{$ B $}
\\ \cline {3-6}
\multicolumn{2}{| l |}{} & \textbf{P} &\textbf{O} & \textbf{X}&\textbf{N} \\
\hline
\multirow{4}{*}{$ A $} & \textbf{P}  & P & O & X & N \\
\cline{2-6}
& \textbf{O} & O & O & N & N \\ \cline{2-6}
& \textbf{X} & X & N & X & N \\ \cline{2-6}
& \textbf{N} & N & N & N & N \\ \hline
\end{tabular}
\begin{tabular}[p]{| l | l | l | l | l | l |}
\hline
\multicolumn{2}{| l |}{\multirow{2}{*}{$ A \vee B $}} & \multicolumn{4}{ c |}{$ B $}
\\ \cline {3-6}
\multicolumn{2}{| l |}{} & \textbf{P} &\textbf{O} & \textbf{X}&\textbf{N} \\
\hline
\multirow{4}{*}{$ A $} & \textbf{P}  & P & P & P & P \\
\cline{2-6}
& \textbf{O} & P & O & P & O \\ \cline{2-6}
& \textbf{X} & P & P & X & X \\ \cline{2-6}
& \textbf{N} & P & O & X & N \\ \hline
\end{tabular}
\begin{tabular}[p]{| l | l | l | l | l | l |}
\hline
\multicolumn{2}{| l |}{\multirow{2}{*}{$ A \rightarrow B $}} & \multicolumn{4}{ c |}{$ B $}
\\ \cline {3-6}
\multicolumn{2}{| l |}{} & \textbf{P} &\textbf{O} & \textbf{X}&\textbf{N} \\
\hline
\multirow{4}{*}{$ A $} & \textbf{P}  & P & O & X & N \\
\cline{2-6}
& \textbf{O} & P & O & P & O \\ \cline{2-6}
& \textbf{X} & P & P & X & X \\ \cline{2-6}
& \textbf{N} & P & P & P & P \\ \hline
\end{tabular}

\caption{Protože Kleeneho trojhodnotová logika už je \uv{zastaralá}, uvádíme si zde
příklad čtyřhodnotové logiky}
\label{table:logika}
\end{table}
\bigskip
%--------------------------------------------------------------------

\pagebreak
%========================== 2. Strana (ALGORITMY)  ================================%

\section{Algoritmy}
\label{section:algoritmy}
Pokud budeme chtít vysázet algoritmus, můžeme použít prostředí \texttt{ algoritm\footnote{Pro nápovědu, jak zacházet s~prostředím\texttt{ algorithm,} můžeme zkusit tuhle stránku: \\
\href{http://ftp.cstug.cz/pub/tex/CTAN/macros/latex/contrib/algorithms/algorithms.pdf}{http://ftp.cstug.cz/pub/tex/CTAN/macros/latex/contrib/algorithms/algorithms.pdf}. }}
nebo \texttt{ algorithm2e\footnote{ Pro \texttt{algorithm2e} zase tuhle: \href{http://ftp.cstug.cz/pub/tex/CTAN/macros/latex/contrib/algorithm2e/doc/algorithm2e.pdf}{http://ftp.cstug.cz/pub/tex/CTAN/macros/latex/contrib/algorithm2e/doc/algorithm2e.pdf}.}}.
Příklad použití prostředí \texttt{algorithm2e} viz Algoritmus~\ref{algorithm:algoritmy2}.\\

\IncMargin{2em}
	\begin{algorithm}[H]
        
		\SetNlSty{}{}{:}
		\SetNlSkip{0.5em}
		\SetAlgoNlRelativeSize{-1}
		\SetInd{0.6em}{0.6em}

		\Indm\Indmm
		\KwIn{$(X_{t - 1}, u_t, z_t)$}
		\KwOut{$X_t$}
		\Indp\Indpp
		\BlankLine
        $ \overline{X_t} = X_t = 0 $ \\
 
        \For{$ k=1 \textrm{\emph{to}} M $}{
          $x_t^{[k]} = \emph{sample\_motion\_model}(u_t, x_{t-1}^{[k]}) $ \\
          $ \omega_t^{[k]} = \emph{measurement\_model}(z_t,x_{t^[k]},m_{t-1}^{[k]} $ \\
          $ m_t^{[k]} = updated\_occupancy\_grid(z_t, x_t^{[k]}, m_{t-1}^{[k]}) $ \\
          $ \overline{X_t} = \overline{X_t} + \langle x_x^{[m]}, \omega_t^{[m]} \rangle $ \\
        }
        
		\For{$ k = 1 \textrm{\emph{ to }} M $}{
			draw $ i $ with probability $ \approx \omega_t^{[i]} $ \\
			add $ \langle x_x^{[k]}, m_t^{[k]} \rangle \textrm{ to } X_t $ \\
		}        
\Return{$ X_t $}
\caption{\textsc{FastSLAM}}
\label{algorithm:algoritmy2}
\end{algorithm}


\section {Obrázky}
Do našich článků můžeme samozřejmě vkládat obrázky. Pokud je obrázkem fotografie,
můžeme klidně použít bitmapový soubor. Pokud by to ale mělo být nějaké schéma nebo
něco podobného, je dobrým zvykem takovýto obrázek vytvořit vektorově.

\begin{figure}[h]
    \centering
        \scalebox{0.4}{
            \includegraphics{img-proj3/etiopan.eps}
            \reflectbox{\includegraphics{img-proj3/etiopan.eps}}
            }
        \caption{Malý Etiopánek a~jeho bratříček}
        \label{figure:etiopanek}
\end{figure}
\pagebreak

Rozdíl mezi vektorovým\,\dots
        
\begin{figure}[h]
    \scalebox{0.4}{\includegraphics{img-proj3/oniisan.eps}}
    \centering
    \caption{Vektorový obrázek}
    \label{figure:vektor}
\end{figure}
\bigskip
\noindent
\dots\, a~bitmapovým obrázkem

\begin{figure}[h]
    \scalebox{0.6}{\includegraphics{img-proj3/oniisan2.eps}}
    \centering
    \caption{Bitmapový obrázek}
    \label{figure:raster}
\end{figure}
\bigskip 
\noindent {se projeví například při zvětšení.}

Odkazy (nejen ty) na obrázky~\ref{figure:etiopanek},~\ref{figure:vektor} a~\ref{figure:raster}, na tabulky~\ref{table:mena} a~\ref{table:logika}~a také na algoritmus~\ref{algorithm:algoritmy2} jsou udělány pomocí křížových odkazů. Pak je ovšem potřeba zdrojový soubor přeložit dvakrát. \par
Vektorové obrázky lze vytvořit i~přímo v~{\LaTeX}u, například pomocí prostředí \texttt{picture.}\par

%=================================== PAINTING IMAGE ======================================%

\begin{landscape}
		\begin{figure}[p]
			\setlength{\unitlength}{1mm}
			\centering
			\begin{picture}(200, 110)
				\linethickness{1pt}
				\put(0, 0){\framebox(200, 110){}}
				\linethickness{1.5mm}
				\put(4,14){\line(1,0){192}}
				\linethickness{0.4mm}
				\put(175, 90){\circle{14}}
				\put(24, 14){\line(0, 0){36}}
				\put(24, 50){\line(1, 0){43}}
				\multiput(67, 45)(56, 0){2}{\line(0, 0){10}}
				\put(67, 55){\line(1, 0){56}}
				\put(123, 47){\line(1, 0){49}}
				\put(172, 47){\line(0, -1){2}}
				\multiput(43, 45)(0, -6){2}{\line(1, 0){142}}
				\multiput(43, 45)(142, 0){2}{\line(0, -1){6}}
				\put(43, 39){\line(1, -1){11}}
				\put(14, 15){\line(0, 1){70}}
				\put(14, 85){\line(1, 0){25}}
				\put(39, 50){\line(0, 1){35}}
				\put(18, 70){\line(1, 0){18}}
				\put(50, 50){\line(4, 1){17}}
				\put(150, 47){\line(-4, 1){27}}
				
				\put(67, 55){\line(6, 1){28}}
				\put(123, 55){\line(-6, 1){28}}
				
				\put(35, 14){\line(0, 0){14}}
				\put(35, 28){\line(1, 0){34}}
				\put(69, 28){\line(3, -1){41}}
				\put(75, 26){\line(0, 0){11}}
				\put(75, 37){\line(1, 0){105}}
				\put(180, 37){\line(0, -1){14}}
				\put(182, 23){\line(-1, 0){97}}
				\put(182, 23){\line(0, -1){9}}
			\end{picture}
			\caption{Vektorový obrázek moderního bydlení vhodného pro 21. století.}
		\end{figure}
	\end{landscape}

\end{document}